% --- Языки и кодировки
\usepackage[T2A]{fontenc}
\usepackage[utf8]{inputenc}
\usepackage[russian]{babel}

% --- Математические ПАКЕТЫ (Логика и команды)
\usepackage{amsmath}   % Нужен для окружений типа align, gather
\usepackage{amsfonts}  % Нужен для ажурных букв типа \mathbb{R}
\usepackage{amssymb}   % Нужен для дополнительных символов
\usepackage{amsthm}    % Нужен для оформления теорем и доказательств

% --- ШРИФТЫ (Внешний вид)
\usepackage{tempora}            % Основной текст — Times New Roman (с кириллицей)
\usepackage[libertine]{newtxmath} % Математика в стиле Times (совместим с amsmath)

% --- Геометрия страницы (стандарт ГОСТ 2.105) ---
\usepackage[left=30mm, right=10mm, top=20mm, bottom=20mm]{geometry}

% --- Настройка интервалов ---
\usepackage{setspace}
\onehalfspacing{} % Полуторный интервал (обычно требуется в дипломах)
\usepackage{indentfirst} % Абзацный отступ в первом параграфе секции

% --- Оглавление ---
\usepackage{tocloft}
\renewcommand{\contentsname}{Содержание}
\renewcommand{\cftsecleader}{\cftdotfill{\cftdotsep}} % Точки в оглавлении
\setcounter{tocdepth}{4} 
\setcounter{secnumdepth}{4}

% --- Графика и таблицы ---
\usepackage{graphicx}
\usepackage{multirow}
\usepackage{longtable}
\usepackage{float}
\usepackage{adjustbox}
\usepackage[inkscapearea=page, inkscapeformat=png]{svg} 
\usepackage{subcaption}

% Пути к папкам с картинками (чтобы не писать img/ везде)
\graphicspath{{img/}}
\svgpath{{img/}}

% Настройка подписей под рисунками
\usepackage{caption}
\captionsetup[figure]{name=Рисунок, labelsep=endash}

% --- Код и алгоритмы ---
\usepackage{algorithm}
\usepackage{algpseudocode}
\usepackage{listings}
\lstset{
  language=Java,
  basicstyle=\small\ttfamily,
  breaklines=true,
  frame=single
}

% --- Библиография ---
\usepackage{csquotes} % Рекомендуется для biblatex
\usepackage[
    backend=biber,     % Используем современный движок biber
    style=gost-numeric,% Стиль по ГОСТу (цифровой)
    sorting=none,      % Сортировка в порядке цитирования (часто требуют в дипломах)
    doi=true,          % Показывать DOI
    url=true,          % Показывать URL
    language=auto,
    autolang=other,
    clearlang=true
]{biblatex}

\addbibresource{bib/references.bib} % Путь к файлу с литературой

% --- Гиперссылки ---
\usepackage[unicode, colorlinks=false, hidelinks]{hyperref}
\usepackage[all]{hypcap}

% --- Дополнительные настройки ---
\usepackage{pdfpages}

\newcounter{intvl} \setcounter{intvl}{3}
\newcounter{otstup} \setcounter{otstup}{0}
\newcounter{contnumeq} \setcounter{contnumeq}{0}
\newcounter{contnumfig} \setcounter{contnumfig}{0}
\newcounter{contnumtab} \setcounter{contnumtab}{1}
\newcounter{pgnum} \setcounter{pgnum}{1}
\newcounter{chapstyle} \setcounter{chapstyle}{1}
\newcounter{headingdelim} \setcounter{headingdelim}{1}
\newcounter{headingalign} \setcounter{headingalign}{0}
\newcounter{headingsize} \setcounter{headingsize}{0}

\AtBeginDocument{\DeclareCaptionSubType{lstlisting}}